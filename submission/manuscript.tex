\documentclass[11pt,]{article}
\usepackage{lmodern}
\usepackage{amssymb,amsmath}
\usepackage{ifxetex,ifluatex}
\usepackage{fixltx2e} % provides \textsubscript
\ifnum 0\ifxetex 1\fi\ifluatex 1\fi=0 % if pdftex
  \usepackage[T1]{fontenc}
  \usepackage[utf8]{inputenc}
\else % if luatex or xelatex
  \ifxetex
    \usepackage{mathspec}
  \else
    \usepackage{fontspec}
  \fi
  \defaultfontfeatures{Ligatures=TeX,Scale=MatchLowercase}
\fi
% use upquote if available, for straight quotes in verbatim environments
\IfFileExists{upquote.sty}{\usepackage{upquote}}{}
% use microtype if available
\IfFileExists{microtype.sty}{%
\usepackage{microtype}
\UseMicrotypeSet[protrusion]{basicmath} % disable protrusion for tt fonts
}{}
\usepackage[margin=1.0in]{geometry}
\usepackage{hyperref}
\hypersetup{unicode=true,
            pdfborder={0 0 0},
            breaklinks=true}
\urlstyle{same}  % don't use monospace font for urls
\usepackage{graphicx,grffile}
\makeatletter
\def\maxwidth{\ifdim\Gin@nat@width>\linewidth\linewidth\else\Gin@nat@width\fi}
\def\maxheight{\ifdim\Gin@nat@height>\textheight\textheight\else\Gin@nat@height\fi}
\makeatother
% Scale images if necessary, so that they will not overflow the page
% margins by default, and it is still possible to overwrite the defaults
% using explicit options in \includegraphics[width, height, ...]{}
\setkeys{Gin}{width=\maxwidth,height=\maxheight,keepaspectratio}
\IfFileExists{parskip.sty}{%
\usepackage{parskip}
}{% else
\setlength{\parindent}{0pt}
\setlength{\parskip}{6pt plus 2pt minus 1pt}
}
\setlength{\emergencystretch}{3em}  % prevent overfull lines
\providecommand{\tightlist}{%
  \setlength{\itemsep}{0pt}\setlength{\parskip}{0pt}}
\setcounter{secnumdepth}{0}
% Redefines (sub)paragraphs to behave more like sections
\ifx\paragraph\undefined\else
\let\oldparagraph\paragraph
\renewcommand{\paragraph}[1]{\oldparagraph{#1}\mbox{}}
\fi
\ifx\subparagraph\undefined\else
\let\oldsubparagraph\subparagraph
\renewcommand{\subparagraph}[1]{\oldsubparagraph{#1}\mbox{}}
\fi

%%% Use protect on footnotes to avoid problems with footnotes in titles
\let\rmarkdownfootnote\footnote%
\def\footnote{\protect\rmarkdownfootnote}

%%% Change title format to be more compact
\usepackage{titling}

% Create subtitle command for use in maketitle
\providecommand{\subtitle}[1]{
  \posttitle{
    \begin{center}\large#1\end{center}
    }
}

\setlength{\droptitle}{-2em}

  \title{}
    \pretitle{\vspace{\droptitle}}
  \posttitle{}
    \author{}
    \preauthor{}\postauthor{}
    \date{}
    \predate{}\postdate{}
  
\usepackage{helvet} % Helvetica font
\renewcommand*\familydefault{\sfdefault} % Use the sans serif version of the font
\usepackage[T1]{fontenc}

\usepackage[none]{hyphenat}

\usepackage{setspace}
\doublespacing
\setlength{\parskip}{1em}

\usepackage{lineno}

\usepackage{pdfpages}

\begin{document}

\vspace*{10mm}

\hypertarget{reintroducing-mothur-10-years-later}{%
\section{Reintroducing mothur: 10 years
later}\label{reintroducing-mothur-10-years-later}}

\vspace{35mm}

Patrick D. Schloss\({^1}\)\({^\dagger}\)

\vspace{40mm}

\(\dagger\) To whom correspondence should be addressed:
\href{mailto:pschloss@umich.edu}{pschloss@umich.edu}

\(1\) Department of Microbiology and Immunology, University of Michigan,
Ann Arbor, MI 48109

\vspace{35mm}

\hypertarget{observation-format}{%
\subsection{Observation format}\label{observation-format}}

\newpage
\linenumbers

\hypertarget{abstract}{%
\subsection{Abstract}\label{abstract}}

75 words

\newpage

\hypertarget{importance}{%
\subsection{Importance}\label{importance}}

150 words

\newpage

3000 words total

Few scientists set out on a nearly two decade long journey with a
specific goal in mind. Often we fail to start a scientific journey
because it looks too hard. Perhaps we get bogged down in all of the
things that could go wrong. Perhaps we go astray from the path because
we find something else that appears more interesting. Every scientist
picks their own path and takes their own forks in the road. From the
outside, it may appear to be a random walk. Nevertheless, these
meandering journeys in are common in science.

At the risk of navel gazing, looking back on our scientific journeys can
be instructive to other scientists who are overwhelmed at the prospect
of looking forward at their careers {[}Lenski and Chemistry guy{]}. By
no means is my personal journey over, but since 2002 I have been on a
journey that I did not realize I was on. Now that the paper introducing
the mothur software package is ten years old and has become the most
cited paper published by \emph{Applied and Environmental Microbiology},
it is worth stepping back and using the development of mothur as a story
that likely has parallels to many other research stories that have taken
time to develop.

I fondly recall preparing a poster for the 2002 meeting of research
groups supported by the NSF-supported Microbial Observatories Program. I
wanted to triumphantly show that I had sequenced more than 600 16S rRNA
gene sequences from a single 0.5-g sample of Alaskan soil. This was
greater sequencing depth than anyone else had achieved for a single
sample. As I was preparing the poster, I walked into the office of Jo
Handelsman, my postdoctoral research advisor, and laid out the outline
for the poster. She asked if I could add one of those ``curvy things'',
a rarefaction curve, to show where I was in sampling the community.
Rarefaction curves and attempts to estimate the taxonomic richness of
soil had become popular because of the simple, but impactful mini-review
by Jennifer Hughes and her colleagues, which introduced the field to
operational taxonomic units (OTUs), rarefaction curves, and richness
estimates {[}DOI: 10.1128/AEM.67.10.4399-4406.2001{]}. I do not recall
whether that poster had a rarefaction curve on it, but her question
primed my career.

\textbf{\emph{Introducing DOTUR and friends.}} When Jo asked me to
generate a rarefaction curve for the poster, the request was not
trivial. How would I bin the sequences into OTUs? Hughes and her
colleagues did it manually for datasets that had fewer than 284
sequences. Although I could possibly do that for my 600 sequences, my
goal was to generate 1,000 sequences from the sample and to repeat that
sampling effort for other samples. I needed something that could be
automated. Furthermore, the software that Hughes used, EstimateS,
required a series of tedious data formatting steps. I had found my first
problem. How would I assign sequences to OTUs and use that data to
estimate the richness and diversity of a sample? The second problem
would be how could I compare the sequences found in one sample to
another sample? The solution to the first problem, DOTUR (Distance-based
OTUs and Richness), took us two years to develop
{[}10.1128/AEM.71.3.1501-1506.2005{]}. DOTUR did two things: given a
matrix describing the genetic distance between pairs of sequences, it
would cluster those sequences into OTUs for any distance threshold to
define the OTUs and then it would use the frequency of each OTU to
calculate a variety of alpha diversity metrics. The solutions to the
second problem would come from our work to develop software including
S-LIBSHUFF {[}2004{]}, SONS (Shared OTUs and Similarity) {[}2006{]}, and
TreeClimber {[}2006{]}. Around the same time, Catherine Lozupone and Rob
Knight were developing their UniFrac tools to compare communities with a
phylogenetic rather than OTU-based approach {[}PMID: 16332807; PMID:
17220268{]}. With these tools, the field of microbial ecology had a
quantitative toolbox for describing and comparing microbial communities.

It is important to remember that we knew there were many problems with
16S rRNA gene sequencing. We knew there were biases from extractions and
amplification {[}XXXX{]}. We knew there were chimeras {[}XXXX{]}.
Getting to the distance matrix required trimming and correcting sequence
errors, aligning them, and finding the most appropriate way to calculate
a pairwise distance. It was also rare to have experimental replication
to perform statistical tests to compare treatment groups. We frequently
used a dataset comparing Scottish soils from from Alison McCaig and
colleagues. This dataset consisted of two experimental groups, each
replicated three times with 45 sequences in each replicate.
Nevertheless, we had excuses and work arounds for these problems that
served our needs. At the time, I felt that the biggest problems were how
to cluster the sequences into OTUs and how to use those clusterings to
test our hypotheses. Along the way we would demonstrate the utility of
such tools to answer questions like where are we in the bacterial
census? How many sequences would it take to see every OTU in that sample
of Alaskan soil? How does the word usage of \emph{Goodnight, Moon}
compare to that of \emph{Portrait of a Lady}? More importantly, 1,900
papers used DOTUR to facilitate their own research questions. Had we
waited to solve all of the problems that plague 16S rRNA gene
sequencing, we would still be waiting.

As we developed these tools, I found a unique niche in microbiology. I
believe that my undergraduate and graduate training as a biological
engineer prepared me to think about research questions from a systems
perspective, to think quantitatively, and to understand the value of
using computer programs to help solve problems. As an undergraduate
engineering student, I learned the Pascal programming language and
promptly forgot much of it as an engineering graduate student. As a
postdoc, I learned the Perl programming language to better understand
how LIBSHUFF, a tool for comparing the structure of two communities,
worked since it was written in Perl {[}DOI:
10.1128/AEM.70.9.5485-5492.2004{]}. After writing my own version of
LIBSHUFF and seeing the speed of the version written in C++ by my
collaborator, Bret Larget, I converted my Perl version of DOTUR into
C++. At the time, the conversion from Perl to C++ seemed like an
academic exercise to learn a new language. My Perl version only took a
minute or so to process the final collection of 1,000 sequences and the
C++ version took seconds. Was that really such a big difference? In
hindsight, as we now process datasets with millions of sequences, the
decision to learn to C++ was critical. The ability to pick up computer
languages to solve problems was enabled by my prior training. It was
also a skill that was virtually unheard of in microbiology. Today
researchers without the ability to program in Python or R are at a
significant disadvantage.

\textbf{\emph{Introducing mothur.}} Shortly after DOTUR was published, I
received an email from Mitch Sogin asking whether DOTUR could handle
more than a million sequences. Without answering his question, I asked
where he found a million sequences. Little did I know that his email
would represent another pivot in the development of these tools. His
group would be the first to use 454 sequencing technology to generate
16S rRNA gene sequences {[}PMID: 16880384{]}. Although mothur could
assign those sequences to OTUs, it was slow and required a significant
amount of RAM. As I left my postdoc to start my independent career
across the state from Sogin's lab at the University of Massachusetts in
Amherst, my plan was to rewrite DOTUR, SONS, S-LIBSHUFF, and TreeClimber
for the new world of massively parallelized sequencing. The new tool was
mothur.

Milling about at a poster session at an ASM General Meeting in New
Orleans, I again ran into Mitch who asked what my plans were for new
tools. I told him that I wanted to make a tool like ARB (a powerful
database tool and phylogenetics package), but for microbial ecology
analysis. His retort was, ``You and what army?'' To that point, I had
written every line of code and been answering many emails from people
asking for help. It would be difficulty, but I needed to learn to let go
and share the development process with someone else. He was right, I
would need an army. That ``army'' ended up being Sarah Westcott who has
worked on the mothur project largely from its inception. Today, mothur
is over 200,000 lines of code and Sarah has touched or written nearly
every line of code. Beyond writing and testing mothur's code base, she
has become a conduit for many learning the tools of microbial ecology by
patiently answering questions via email and the package's discussion
forum. The community and I are lucky that Sarah has stayed with the
project for more than a decade. To be honest, such dependency on a
single person makes the project brittle. In hindsight, it would have
been better to have developed mothur with more of an ``army'' or team so
that there is overlap in people's understanding of how mothur works.
Although such an arrangement might work in a software engineering firm,
it is not practical in an academic setting where funding is limited for
developers of free, open source software packages. There are certainly
projects that make this work, but they are rare.

\textbf{\emph{Challenges of making open source count.}} Anyone can post
code to GitHub with a permissive license and claim to be an open source
software developer. Far more challenging is engaging the target
community to make contributions to that code. Frankly, we have struggled
to expand the number of people that make contributions to the mothur
code base. One challenge we face is that if we looked to third parties
to contribute code to mothur, they would need to know C++. Given the
paucity of microbiologists with any programming skills, expecting that
community to provide contributors that can write code in a syntax that
prizes execution efficiency over developer efficiency was not likely. In
contrast, the QIIME development team could be more distributed because
their code base was primarily written in Python, which prizes developer
efficiency over execution efficiency and exists as a series of wrappers
to execute other developers' code. These choices resulted in many
tradeoffs that have impacted ease of installation, usability, execution
speed, and flexibility. If we were offered a grant to rewrite mothur, we
would likely rewrite it as an R package that leaned heavily on the R
language's C++ packages. Of course, such choices are always best in
hindsight and when we started developing mothur, the ability to
interface between scripting languages like R and Python and C++ code was
not as well developed as it is today, For example, the modern version of
the Rcpp package was first released in 2009 and its popularity was not
immediate. Again, the development of mothur has been a product of the
environment that it was created in. Although these decisions have
largely had positive outcomes, there have been tradeoffs that caused us
to sacrifice other goals.

Beyond contributing to the mothur code base, we sought out other ways to
include the community as developers. The paper describing mothur
included \textbf{N} co-authors, all but three (Schloss, Ryabin, and
Westcott) responded to a call to provide a wiki page that described how
they used an early version of mothur to analyze a data set. Our vision
was that authors might use the mothur wiki to document reproducible
workflows for papers using mothur but to also provide instructional
materials for other seeking to adapt mothur for their uses. Again, this
vision was a product of the environment. IPython notebooks (2011) and R
markdown (2012) would not be developed until later. Unfortunately, once
the incentive of co-authorship was removed, researchers stopped
contributing their workflows to the wiki. Part of the difficulty of
recruiting wiki contributors was a perception by some that the wiki was
not a community resource. For example, I would frequently receive emails
from people telling me that there was a typo on a specific page when the
intention was that they could correct the typos without my input. We
have been more successful in soliciting input and contributions from the
user community through the mothur discussion forum and GitHub issue
tracker. As mothur has matured, we have been dependent on the user
community to use these resources to tell us what features they would
like to see included in mothur. The user community also tells us where
our documentation is confusing. Often we can count on people not
directly affiliated with mothur to provide instruction and their own
experience to other users. We are constantly trying to recruit our
``army'' and are happy to take any contributions we can. Whether the
contributions are to the code base, discussion forum, or suggestions for
new tools, these contributions have been invaluable to the growth and
popularity of mothur.

\textbf{\emph{Failed experiments.}} If we never failed, we would not be
trying hard enough. Over the past decade we have tried a number of
experiments to improve the usability and utility of mothur. One of our
first experiments was to use mothur to generate standard vector graphic
(SVG)-formatted files of heatmaps and venn diagrams depicting the
overlap between microbial communities. I quickly realized that I would
never put a mothur-generated figure into a manuscript I wrote. Such
visuals require far too much customization to be publication-quality.
Although QIIME has incorporated visualization tools through the Emperor
package, the challenge of users taking default values has downsides as
ordinations with black background or publishing 3-D ordinations in a 2-D
medium litter the literature. Instead, we have encouraged users to use R
packages to visualize mothur-generated results using the minimalR
instructional materials. A second experiment we attempted was to create
a graphical user interface (GUI) for running mothur. Forcing users to
interact with mothur through the command line has been a significant
hurdle for many. Unfortunately, the development effort required to
create and maintain a GUI is significant and there is limited funding
for such efforts. The newest version of QIIME (version 2) has emphasized
interaction with the tools through a GUI and it remains to be seen how
this experiment will go. Another downside of using a GUI is that there
is a risk that reproducibility will suffer if users do not have a
mechanism to document their mouse clicks. This documentation is explicit
in mothur as all commands and output is recorded in a logfile. Given the
heightened recent focus on reproducibility we have extended significant
effort in developing instructional materials teaching users how to
organize, document, and execute reproducible pipelines that allow a user
to go from raw sequence data to a compiled manuscript with figures
through the Riffomonas project. Finally, we collaborated with
programmers through Google Summer of Code to develop commands in mothur
to implement the random forest and SVM machine learning algorithms.
Similar to the challenges of developing attractive visuals, fitting the
algorithms' hyperparameters, testing, and deploying the resulting models
requires a significant amount of customization. Furthermore, this is an
active area of research where methods are still being developed and
improved. Thankfully, there are numerous R and Python packages that do a
better job of developing these models. Again, we have put our efforts
into developing instructional materials that mothur users can use to fit
such models to their data. In each of our ``failed'' experiments, the
real problems were straying from what mothur does well and failing to
grasp what we really wanted the innovation to do. In hindsight, our
solution to these failure has been to provide tutorials to as a conduit
between mothur and their goals.

\textbf{\emph{Competition is good and healthy.}}

\begin{itemize}
\tightlist
\item
  A lot of people have been working in this space - e.g.~clustering
  algorithms
\item
  Approaches to developing software have differed (e.g.~web-based
  (greengenes/RDP), boutique (Quince), wrappers (QIIME))
\end{itemize}

What are we proud of? * Open source packages * Platform independent *
Development of open instructional materials * Data driven development of
methods * Database independent - allows people outside of our field to
use mothur * Reproducibility * Helped create standards - when something
is this popular, you force people to use a certain suite of tools and
methods

The future * 16S rRNA gene sequencing is not dead * Always improving
sequencing methods bring new problems * Larger datasets, need a
replacement for naive open and closed reference clustering * Lingering
controversies - singletons, ASVs, contamination

Longterm funding * The development of mothur was enabled by initial
funding from a grant to Sogin from the Sloan Foundation to support his
VAMPS (Visualization and Analysis of Microbial Population Structures)
initiative. * NSF * NIH * Bootstrapping from other projects

Red Queen

\newpage

\hypertarget{acknowledgements}{%
\subsection{Acknowledgements}\label{acknowledgements}}

\newpage

\hypertarget{references}{%
\subsection{References}\label{references}}

25 references

\hypertarget{refs}{}

\newpage

\textbf{Figure 1. Caption caption caption.} Footnotes footnotes
footnotes.


\end{document}
