\documentclass[11pt,]{article}
\usepackage{lmodern}
\usepackage{amssymb,amsmath}
\usepackage{ifxetex,ifluatex}
\usepackage{fixltx2e} % provides \textsubscript
\ifnum 0\ifxetex 1\fi\ifluatex 1\fi=0 % if pdftex
  \usepackage[T1]{fontenc}
  \usepackage[utf8]{inputenc}
\else % if luatex or xelatex
  \ifxetex
    \usepackage{mathspec}
  \else
    \usepackage{fontspec}
  \fi
  \defaultfontfeatures{Ligatures=TeX,Scale=MatchLowercase}
\fi
% use upquote if available, for straight quotes in verbatim environments
\IfFileExists{upquote.sty}{\usepackage{upquote}}{}
% use microtype if available
\IfFileExists{microtype.sty}{%
\usepackage{microtype}
\UseMicrotypeSet[protrusion]{basicmath} % disable protrusion for tt fonts
}{}
\usepackage[margin=1.0in]{geometry}
\usepackage{hyperref}
\hypersetup{unicode=true,
            pdfborder={0 0 0},
            breaklinks=true}
\urlstyle{same}  % don't use monospace font for urls
\usepackage{graphicx,grffile}
\makeatletter
\def\maxwidth{\ifdim\Gin@nat@width>\linewidth\linewidth\else\Gin@nat@width\fi}
\def\maxheight{\ifdim\Gin@nat@height>\textheight\textheight\else\Gin@nat@height\fi}
\makeatother
% Scale images if necessary, so that they will not overflow the page
% margins by default, and it is still possible to overwrite the defaults
% using explicit options in \includegraphics[width, height, ...]{}
\setkeys{Gin}{width=\maxwidth,height=\maxheight,keepaspectratio}
\IfFileExists{parskip.sty}{%
\usepackage{parskip}
}{% else
\setlength{\parindent}{0pt}
\setlength{\parskip}{6pt plus 2pt minus 1pt}
}
\setlength{\emergencystretch}{3em}  % prevent overfull lines
\providecommand{\tightlist}{%
  \setlength{\itemsep}{0pt}\setlength{\parskip}{0pt}}
\setcounter{secnumdepth}{0}
% Redefines (sub)paragraphs to behave more like sections
\ifx\paragraph\undefined\else
\let\oldparagraph\paragraph
\renewcommand{\paragraph}[1]{\oldparagraph{#1}\mbox{}}
\fi
\ifx\subparagraph\undefined\else
\let\oldsubparagraph\subparagraph
\renewcommand{\subparagraph}[1]{\oldsubparagraph{#1}\mbox{}}
\fi

%%% Use protect on footnotes to avoid problems with footnotes in titles
\let\rmarkdownfootnote\footnote%
\def\footnote{\protect\rmarkdownfootnote}

%%% Change title format to be more compact
\usepackage{titling}

% Create subtitle command for use in maketitle
\providecommand{\subtitle}[1]{
  \posttitle{
    \begin{center}\large#1\end{center}
    }
}

\setlength{\droptitle}{-2em}

  \title{}
    \pretitle{\vspace{\droptitle}}
  \posttitle{}
    \author{}
    \preauthor{}\postauthor{}
    \date{}
    \predate{}\postdate{}
  
\usepackage{helvet} % Helvetica font
\renewcommand*\familydefault{\sfdefault} % Use the sans serif version of the font

%\usepackage{mathpazo} % Palatino font
%\renewcommand*\familydefault{\rmdefault} % Use the roman version of the font

\usepackage[T1]{fontenc}

\usepackage[none]{hyphenat}

\usepackage{setspace}
\doublespacing
\setlength{\parskip}{1em}

\usepackage{lineno}

\usepackage{pdfpages}

\begin{document}

\vspace*{10mm}

\hypertarget{reintroducing-mothur-10-years-later}{%
\section{Reintroducing mothur: 10 years
later}\label{reintroducing-mothur-10-years-later}}

\vspace{35mm}

Patrick D. Schloss\({^1}\)\({^\dagger}\)

\vspace{40mm}

\(\dagger\) To whom correspondence should be addressed:
\href{mailto:pschloss@umich.edu}{pschloss@umich.edu}

\(1\) Department of Microbiology and Immunology, University of Michigan,
Ann Arbor, MI 48109

\vspace{35mm}

\hypertarget{mini-review}{%
\subsubsection{Mini-review}\label{mini-review}}

\newpage
\linenumbers

\hypertarget{abstract}{%
\subsection{Abstract}\label{abstract}}

250 words

\newpage

\hypertarget{importance}{%
\subsection{Importance}\label{importance}}

150 words

\newpage

Few scientists set out on a nearly two decade-long journey with a
specific goal in mind. Often we fail to start a scientific journey
because it looks too hard. Perhaps we get discouraged by all of the
things that could go wrong. Maybe we stray from the path because we find
something that is more interesting. Every scientist picks their own path
and takes their own forks in the road. From the outside, it may appear
to be a random walk. Nevertheless, these meanderings are common in
science.

Looking back on scientific journeys can be instructive to others who are
overwhelmed at the prospect of looking forward at their careers (1--5).
By no means is my personal journey over, but since 2002 I have been on a
journey that I did not realize I was on. Now that the paper introducing
the mothur software package is ten years old and has become the most
cited paper published by \emph{Applied and Environmental Microbiology}
(7), it is worth stepping back and using the continued development of
mothur as a story that has parallels to many other research stories.

I fondly recall preparing a poster for the 2002 meeting of research
groups supported by the NSF-supported Microbial Observatories Program. I
wanted to triumphantly show that I had sequenced more than 600 16S rRNA
gene sequences from a single 0.5-g sample of Alaskan soil. This was
greater sequencing depth than anyone else had achieved for a single
sample. As I was preparing the poster, I walked into the office of Jo
Handelsman, my postdoctoral research advisor, and laid out the outline
for the poster. She asked if I could add one of those ``curvy things'',
a rarefaction curve, to show where I was in sampling the community.
Rarefaction curves and attempts to estimate the taxonomic richness of
soil had become popular because of the impactful review by Jennifer
Hughes and her colleagues (8). Their seminal paper introduced the field
to operational taxonomic units (OTUs), rarefaction curves, and richness
estimates. I do not recall whether my poster had a rarefaction curve on
it, but Jo's question primed my career.

\textbf{\emph{Introducing DOTUR and friends.}} When Jo asked me to
generate a rarefaction curve for the poster, the request was not
trivial. How would I bin the sequences into OTUs? Hughes and her
colleagues did it manually and with fewer than 300 sequences. Although I
could possibly do that for my 600 sequences, my goal was to generate
1,000 sequences from the sample and to repeat that sampling effort for
other samples. I needed something that could be automated. Furthermore,
the software that Hughes used, EstimateS (4), required a series of
tedious data formatting steps to perform the analyses we were interested
in performing. I had found my first problem. How would I assign
sequences to OTUs and use that data to estimate the richness and
diversity of a sample? The second problem would involve comparing the
abundance of OTUs found in one sample to another sample. The solution to
the first problem, DOTUR (Distance-based OTUs and Richness), took us two
years to develop (9). DOTUR did two things: given a matrix quantifying
the genetic distance between pairs of sequences, it would cluster those
sequences into OTUs for any distance threshold to define the OTUs and
then it would use the frequency of each OTU to calculate a variety of
alpha diversity metrics. The solutions to the second problem would come
from our work to develop software including \(\int\)-LIBSHUFF (10), SONS
(Shared OTUs and Similarity) (11), and TreeClimber (12). Around the same
time, Catherine Lozupone and Rob Knight were developing their UniFrac
tools to compare communities with a phylogenetic rather than OTU-based
approach (13, 14). With these tools, the field of microbial ecology had
a quantitative toolbox for describing and comparing microbial
communities. Along the way Jo and I would demonstrate the utility of
such tools to answer questions like how many OTUs were there in that
sample of Alaskan soil and how many sequences were needed to sample each
of those OTUs (15)? Where were we in the global bacterial census (16)?
How does the word usage of \emph{Goodnight, Moon} compare to that of
\emph{Portrait of a Lady} and more importantly how is this relevant to
microbial ecology (17)? Most edifying were the more than 2,400 papers
that used DOTUR, SONS, TreeClimber, or \(\int\)-LIBSHUFF to facilitate
their own research questions (Web of Science, 10/1/2019). Had we waited
to solve all of the problems that plagued 16S rRNA gene sequencing, we
would still be waiting.

It is important to remember that we knew there were many problems with
16S rRNA gene sequencing. We knew there were biases from extractions and
amplification (18--23). We knew there were chimeras (24--27). We knew
that bacteria varied in their \emph{rrn} copy number. Generating a
distance matrix was a prerequisite to using my tools. This wasn't
trivial, but by cobbling together other tools it was possible. We would
assemble, trim and and correct Sanger sequence reads using Chromas or
STADEN (28), align the sequences using ClustalW (29) or ARB (30), check
for chimeras using partial treeing or Bellerephon (27), and calculate a
pairwise distance matrix using DNADIST from the PHYLIP package (31). At
the time, we knew that we only had a loose concept of a species based on
these distances (32). We hoped that an OTU defined as a group of
sequences more than 97\% similar to each other would be a biologically
meaningful unit regardless of whether it fit our notion of a bacterial
species. At the time, I felt that the biggest problems that I could
solve were how to cluster the sequences into OTUs and how to use those
clusterings to test our hypotheses. The only tool available at the time
that automated the clustering step was FastGroup, which implemented a
greedy version of the single linkage algorithm (33). The high cost of
sequencing was also an impediment to experimentation and analysis in
microbial ecology. It was rare for a study design to have experimental
replicates so that one could perform a statistical test to compare
treatment groups. For example, in our testing we frequently used a
dataset comparing Scottish soils from from Alison McCaig and colleagues
(34). This dataset consisted of two experimental groups, each replicated
three times with 45 sequences per replicate. Although great focus has
been placed on the depth of sampling afforded by 454 and Illumina
sequencing, the true benefit of the modern sequencing platforms is the
ability to affordably sequence a large number of technical and
biological replicates. In spite of the many technical challenges, we had
excuses and heuristics to solve problems that served our needs. It is
telling that a recent review of ``best practices'' in generating and
analyzing 16S rRNA gene sequences shows that we still have not solved
many of these issues and that we have even identified additional
problems (35).

As we developed these tools, I found a unique niche in microbiology. My
undergraduate and graduate training as a biological engineer prepared me
to think about research questions from a systems perspective, to think
quantitatively, and to understand the value of using computer programs
to help solve problems. As an undergraduate student, I learned the
Pascal programming language and promptly forgot much of it as a graduate
student and learned MatLab in its place. As a postdoc, I learned the
Perl programming language to better understand how LIBSHUFF, a tool for
comparing the structure of two communities, worked since it was written
in Perl (36). After writing my own version of LIBSHUFF,
\(\int\)-LIBSHUFF and seeing the speed of the version written in C++ by
my collaborator, Bret Larget, I converted my Perl version of DOTUR into
C++. At the time, the conversion from Perl to C++ seemed like an
academic exercise to learn a new language. My Perl version only took a
minute or so to process the final collection of 1,000 sequences and the
C++ version took seconds. Was that really such a big difference? In
hindsight, as we now process datasets with millions of sequences, the
decision to learn to C++ was critical. The ability to pick up computer
languages to solve problems was enabled by my prior training. It was
also a skill that was virtually unheard of in microbiology. Today,
researchers without the ability to program are at a significant
disadvantage (37).

\textbf{\emph{Introducing mothur.}} Shortly after DOTUR was published, I
received an email from Mitch Sogin, a scientist at the Marine Biology
Laboratory (Woods Hole, MA), who asked whether DOTUR could handle more
than a million sequences. Without answering his question, I asked where
he found a million sequences. Little did I know that his email would
represent another pivot in the development of these tools. His group
would be the first to use 454 sequencing technology to generate 16S rRNA
gene sequences (38). Although DOTUR could assign those sequences to
OTUs, at the scale of millions of sequences, it was slow and required a
significant amount of RAM. As I left my postdoc to start my independent
career across the state from Sogin's lab at the University of
Massachusetts in Amherst, my plan was to rewrite DOTUR, SONS,
\(\int\)-LIBSHUFF, and TreeClimber for the new world of massively
parallelized sequencing. The new tool would become mothur.

Milling about at a poster session at an ASM General Meeting in New
Orleans, I ran into Mitch who asked what my plans were for my new lab. I
told him that I wanted to make a tool like ARB, a powerful database tool
and phylogenetics package (30), but for microbial ecology analysis. His
retort was, ``You and what army?'' Up to that point, I had written every
line of code and been answering many emails from people asking for help.
He was right, I would need an army. It would be difficult, but I needed
to learn to let go and share the development process with someone else.
My ``army'' ended up being Sarah Westcott who has worked on the mothur
project from its inception. Today, mothur is over 200,000 lines of code
and Sarah has touched or written nearly every line of it. Beyond writing
and testing mothur's code base, she has become a conduit for many who
are trying to learn the tools of microbial ecology. She patiently
answers questions via email and on the package's discussion forum
(\url{https://forum.mothur.org}). The community and I are lucky that
Sarah has stayed with the project for more than a decade. To be honest,
such dependency on a single person makes the project brittle. In
hindsight, it would have been better to have developed mothur with more
of an ``army'' or team so that there is overlap in people's
understanding of how mothur works. Although a distributed team approach
might work in a software engineering firm, it is not practical in most
academic environments where there is limited funding. There are
certainly projects that make this work, but they are rare.

\textbf{\emph{Competition has been good and healthy.}} mothur has not
been developed in a vacuum and it does not have a monopoly within the
field. As indicated above, each of our decisions were made in the
historical context of the field and with constant pressure from others
developing their own tools for analyzing 16S rRNA gene sequence data.
Competition has been good for mothur and for the field.

From the beginning there have been online tools available at the
Ribosomal Database Project (RDP) (39), greengenes (40), and SILVA (41).
These allowed users a straightforward method of comparing their data to
those collected in a database. There are two primary downsides to these
tools. First, researchers running the online tool must pay the
computational expenses leading to slow process times when hardware
becomes outdated and when numerous users simultaneously attempt their
analysis. Eventually this limitation would result in the termination of
the greengenes website. Second, these platforms provide a
one-size-fits-all analysis. These tools only allow a user to analyze 16S
and in some cases 18S rRNA gene sequences. If a user sequences a
different gene, then the tool will not serve them. These observations
resulted in two design goals we have had with mothur: bringing the
analysis to a user's computer and separating a tool from a specific
database. For example, we commonly use a sequence alignment method that
was originally developed for greengenes (42), but use a SILVA-based
reference alignment because its superior quality (43, 44). In addition,
we offer the \text{na\"ive} Bayesian classifier developed by the RDP
(45) and allow users to train it to any database they want, including
customized databases. In both examples, users can align or classify
non-rRNA gene sequence data. As the bioinformatics tools have matured,
both the RDP and SILVA offer integrated pipelines for analyzing large
datasets, albeit in one-size-fits-all black box implementations.

With the growth in popularity of 16S rRNA gene sequencing there has
naturally been an expansion in the number of people developing tools to
analyze these data. Months after the paper describing mothur was
published, the paper describing QIIME was published (46). Over the past
10 years, many have attempted to create analogies comparing the two
programs: Pepsi vs Coke, Apple vs Windows, etc. It is never clear which
software is which brand and whether the comparisons are meant as a
complement or an insult. Regardless, both programs are very popular.
From my perspective, most of the differences are cosmetic. QIIME is
effectively a bundle of scripts to run other developers' software. For
example, with QIIME (through version 1.9.1), it was even possible to run
mothur through QIIME. One can also run the \text{na\"ive} Bayesian
classifier through QIIME using the original code developed by the RDP.
This caused great frustration for many users because there were numerous
software dependencies that had to be installed. Although the QIIME
developers would go on to create virtual machines and use packaging
tools to simplify installation, these fixes required sophistication by
users who we knew struggled with the basics of navigating a command
line. In contrast, when a user runs mothur, they are running mothur. The
\text{na\"ive} Bayesian classifier code that is in mothur is a rewritten
version of the original code. When we rewrite someone's software we do
it with an eye to improving performance, access, and utility for non-16S
rRNA gene sequence data. For example, while 454 data was popular,
PyroNoise was an effective tool for denoising flowgram data (47).
Running the original code required a large Linux computer cluster and
knowledge of bash and Perl scripting. When we rewrote the code for
mothur, we made it accessible to people using any operating system with
a simple command interface (i.e.~trim.flows and shhh.flows). Our
approach requires significant developer effort, but saves considerable
user effort. As this benefit is multiplied across thousands of projects,
the savings to users has been considerable.

Beyond the large packages like mothur and QIIME, there has been
significant growth in stand alone software tools for sequence curation
(e.g.~PyroNoise (47), PANDAseq (48), DADA2 (49)), chimera checking
(e.g.~UCHIME (50), ChimeraSlayer (51), Perseus (52)), and clustering
(e.g.~USEARCH (53), VSEARCH (54), Swarm (55)). Where possible and when
warrented, we have implemented many of these algorithms directly into
mothur. We have also used this diversity of methods to perform
head-to-head comparisons. Most notable is the area of clustering
algorithms where there have been a large number of algorithms developed
without an obvious method to objectively compare them (56--59). We
applied an objective metric, the Matthew's Correlation Coefficient
(MCC), to evaluate numerous algorithms for clustering sequences into
OTUs. By performing this type of analysis, we were able to objectively
compare the algorithms, make recommendations to the field, and develop
new algorithms that outperformed the existing algorithms. Beyond
evaluating clustering algorithms, we have also evaluated methods of
denoising sequence data (60--62), assessed reference alignments (43,
44), considered the importance of incorporating secondary structure
information in alignments (63), quantified the variation along the 16S
rRNA gene (44), and compared the statistical hypotheses tested by
commonly used tools (64). We have embraced the competition and diversity
of all methods being used to analyze amplicon data as an opportunity to
identify the strengths and weaknesses of the methods in an attempt to
make recommendations to other researchers.

\textbf{\emph{mothur's core principles.}} As mothur has evolved with the
needs of the community, several core principles have emerged that direct
its development. First, mothur is a free, open source software package.
This has been critical in shaping the direction of mothur. We were
content for mothur to be an improved combination of DOTUR and SONS and
leverage existing tools for other steps. Yet, when we learned that the
code for NAST, the algorithm behind the greengenes's aligner (42), was
not open source or publicly available. Similarly, although the SINA
aligner was available through ARB and the SILVA website performed well
it was closed source. Because the ARB implementation did not scale to
large datasets, researchers were left to pay a processing fee to the
SILVA website to align sequences. Thus, we realized that such an
important tool needed to be opened to the community (43). More recently,
the rejection of closed source, commercial tools such as USEARCH can be
seen by the broader adoption of its open source, free competitor,
VSEARCH, within the microbial ecology community (53, 54). Related to
insuring that mothur's code is open source, our second core principle is
that we maintain transparency to our users. Perhaps a user does not need
to interrogate every line of code, but they need to understand what is
happening. Many programs including online workflows encapsulate large
elements of a pipeline in a single command. In contrast, mothur forces
the user to specify each step of the pipeline. Although the former
approach makes an analysis easier for a beginner, it potentially stifles
users that need greater control or understanding of the assumptions at
each step. This control over the pipeline has made it easier for
researchers to customize databases or adapt the pipeline to analyze
non-16S rRNA gene sequence data. Third, as I mentioned above, there has
been a plethora of methods proposed for generating amplicon sequence
data, and curating, aligning, checking for chimeras, classifying, and
clustering the data. I am proud of the data-driven approach we have
taken to comparing these methods. A description of a new method is of
limited value if it is not benchmarked against other methods or control
datasets. Through this core principle and mothur's large reach into the
community, we have helped to develop standards in the analysis of 16S
rRNA gene sequence data. Fourth, a focus on enabling reproducibility has
always been central to the functionality of mothur. From the beginning,
mothur's logfiles have represented a transcript of the user's command
and outputs. When researchers were reluctant to submit sequence data to
the Sequence Read Archive (SRA), we worked with the SRA developers to
create a mothur command (make.sra) to create the package to submit
sequence data through a special mothur portal. A more ambitious project
had its seed on April 1, 2013 when we announced a new ``function'' in
mothur: write.paper. The new command required that the user provide a
454 sff file and a journal title or impact factor. With this
information, mothur would generate a manuscript. This April Fools' Day
joke was poking fun at software that provided an analysis black box but
also at many users' sentiments that data analysis should be so cut and
dry. A few years later, we revisited this concept in the scope of
reproducibility. Why not explicitly script an analysis from downloading
data from the SRA through the rendering of a manuscript ready for
submission? This idea gave rise to the development of the Riffomonas
reproducible research tutorial series that enables researchers to write
their own version of write.paper (65). Perhaps the most important core
principle is that my research group uses mothur to analyze the data we
generate. We ``eat our own dogfood''. This has proven critical as it
again represents transparency and hopefully provides confidence to
mothur's users that we are not making recommendations that we do not
follow ourselves.

\textbf{\emph{Challenges of making open source count.}} Anyone can post
code to GitHub with a permissive license and claim to be an open source
software developer. Far more challenging is engaging the target
community to make contributions to that code. Frankly, we have struggled
to expand the number of people that make contributions to the mothur
code base. One challenge we face is that if we looked to third parties
to contribute code to mothur, they would need to know C++. Given the
paucity of microbiologists with skills programing in a compiled language
like C++, expecting that community to provide contributors that can
write code in a syntax that prizes execution efficiency over developer
efficiency was not likely. In contrast, the QIIME development team could
be more distributed because their code base was primarily written in
Python, which prizes developer efficiency over execution efficiency.
QIIME is a series of wrappers that allow users to execute other
developers' code making the use of a scripting language like Python
attractive. Their choices resulted in many tradeoffs that have impacted
ease of installation, usability, execution speed, and flexibility. If we
were offered funding to rewrite mothur, we would likely rewrite it as an
R package that leaned heavily on the R language's C++ interface
packages. Of course, such choices are always best in hindsight. Yet,
when we started developing mothur, the ability to interface between
scripting languages like R and Python and C++ code was not as well
developed as it is today. For example, the modern version of the Rcpp
package was first released in 2009 and its popularity was not immediate
(66). Again, the development of mothur has been a product of the
environment that it was created in. Although these decisions have
largely had positive outcomes, there have been tradeoffs that caused us
to sacrifice other goals.

Beyond contributing to the mothur code base, we sought out other ways to
include the community as developers. The paper describing mothur
included 15 co-authors, all but three (Schloss, Westcott, and Ryabin)
responded to a call to provide a wiki page that described how they used
an early version of mothur to analyze a data set. Our vision was that
authors might use the mothur wiki to document reproducible workflows for
papers using mothur but to also provide instructional materials for
other seeking to adapt mothur for their uses
(\url{https://www.mothur.org/wiki}). Unfortunately, once the incentive
of co-authorship was removed, researchers stopped contributing their
workflows to the wiki. Again, this vision and the lack of the
community's adoption of wikis as a mechanism for reporting workflows was
a product of the environment. Although wikis were popular in the late
2000's, they lacked the ability to directly execute the commands that
researchers reported. Such technology would not be possible until the
creation of IPython notebooks (2011) and R markdown (2012). Another
problem with the wiki approach was that potential contributors did not
see the wiki as a community resource. I frequently received emails from
scientists telling me that there was a typo on a specific page when the
intention was that they could correct the typos without my input. We
have been more successful in soliciting input and contributions from the
user community through the mothur discussion forum and GitHub-based
issue tracker. As mothur has matured, we have been dependent on the user
community to use these resources to tell us what features they would
like to see included in mothur and where the documentation is confusing
or incomplete (\url{https://forum.mothur.org}). Often we can count on
people not directly affiliated with mothur to provide instruction and
their own experience to other users on the forum. We are constantly
trying to recruit our ``army'' and are happy to take any contributions
we can. Whether the contributions are to the code base, discussion
forum, or suggestions for new tools, these contributions have been
invaluable to the growth and popularity of mothur.

\textbf{\emph{Failed experiments.}} If we never failed, we would not be
trying hard enough. Over the past decade we have tried a number of
experiments to improve the usability and utility of mothur. One of our
first experiments was to use mothur to generate standard vector graphic
(SVG)-formatted files of heatmaps and Venn diagrams depicting the
overlap between microbial communities. Such visuals were helpful for
exploring or data; however, I quickly realized that I would never put a
mothur-generated figure into a manuscript I wrote. Such visuals require
far too much customization to be publication-quality. Although QIIME has
incorporated visualization tools through the Emperor package (67), the
challenge of users taking default values has downsides as ordinations
with black background or publishing 3-D ordinations in a 2-D medium
litter the literature. Instead, we have encouraged users to use R
packages to visualize mothur-generated results using the minimalR
instructional materials that I have developed
(\url{http://www.riffomonas.org/minimalR/}). A second experiment was the
creation of a graphical user interface (GUI) for running mothur. Forcing
users to interact with mothur through the command line has been a
significant hurdle for many. Unfortunately, the development effort
required to create and maintain a GUI is significant and there is
limited funding for such efforts. The newest version of QIIME (starting
with version 2.0.0) has emphasized interaction with the tools through a
GUI (68) and the related QIITA project offers a web-based GUI (69). It
remains to be seen how this experiment will go. Another downside of
using a GUI is that there is a risk that reproducibility will suffer if
users do not have a mechanism to document their mouse clicks.
Documentation of commands and parameter values is explicit in mothur as
users can provide the software a file with a list of commands and all
commands and output are recorded in a logfile. Given the heightened
focus on reproducibility in recent years (70), we have extended
significant effort in developing instructional materials teaching users
how to organize, document, and execute reproducible pipelines that allow
a user to go from raw sequence data to a compiled manuscript with
figures through the Riffomonas project (65). A final example of a failed
experiment was a collaboration with programmers through Google Summer of
Code to develop commands in mothur that ran the random forest and SVM
machine learning algorithms. Similar to the challenges of developing
attractive visuals, fitting the algorithms' hyperparameters, testing,
and deploying the resulting models require a significant amount of
customization. Furthermore, machine learning is an active area of
research where methods are still being developed and improved.
Thankfully, there are numerous R and Python packages that do a better
job of developing these models (71, 72). Again, we have put our efforts
into developing instructional materials that mothur users can use to fit
such models to their data. In each of our ``failed'' experiments, the
real problems were straying from what mothur does well and failing to
grasp what we really wanted the innovation to do.

\textbf{\emph{The future.}} I will continue to develop mothur for as
long as other researchers find it useful. One challenge of such a plan
is maintaining the funding to support its development. The development
of mothur was initially enabled by a subcontract from a Sloan Foundation
grant to Mitch Sogin to support his VAMPS (Visualization and Analysis of
Microbial Population Structures) initiative. We used that seed funding
to secure an NSF grant and then a grant from NIH for tool development as
part of their Human Microbiome Project. Since that project expired in
2013, we have not had funding to specifically support mothur's
development. I have been fortunate to have start-up and discretionary
funds generated from other projects to help support mothur. Although
there is funding for new tools, there appears to be little appetite by
funders to support existing tools. Emblematic of this was the NIH
program, Big Data To Knowledge (BD2K), which solicited proposals through
the program announcement ``Extended Development, Hardening and
Dissemination of Technologies in Biomedical Computing, Informatics and
Big Data Science (PA-14-156)''. This opportunity appeared perfect,
except that the National Institute of Allergy and Infectious Diseases
(NIAID), the primary supporter of microbiome research at NIH, did not
participate in the announcement. Tools like mothur are clearly
successful, but need funding mechanisms to continue to mature and
support the needs of the research community.

As with anything in science, methods become \text{pass\'e}. When we
first developed mothur, T-RFLP and DGGE were still commonly used. Today
it would be hard to argue that data from those methods meaningfully
advance a study relative to what one could get using 16S rRNA gene
sequence data. Looking forward, many want to claim that amplicon
sequencing is today's DGGE. They claim that researchers should instead
move on to shotgun metagenomic sequencing. It is important to note that
the two methods answer fundamentally different questions. 16S rRNA gene
sequence data describes the taxonomic composition and metagenomic
sequence data tells a researcher about the functional potential and
genetic diversity of a community. Both tools provide important
information but they cannot easily replace each other. Although
metagenomic data does provide highly resolved taxonomic information, the
limit of detection is at least an order of magnitude higher than that of
amplicon data. For example, we analyzed 10,000 16S rRNA sequences from
each of about 500 subjects (73). We can think of this as representing
about 1,000,000 genome equivalents (10,000 16S rRNA genes/subject x 500
subjects / 5 16S rRNA gene sequences/genome). Assuming a genome is 4
Mbp, this would represent a sequencing depth of 4 Tbp. Although such a
sequencing effort is technically possible, the cost of such an endeavor
would be considerable and unlikely to be pursued by most researchers. We
estimate that generating and sequencing the libraries at the University
of Michigan sequencing core would cost approximately \$150 per library.
The parallel 16S rRNA gene sequences data would cost approximately \$8
per library. Furthermore, analyzing such a large dataset with an
approach that captures the full genetic diversity of the community would
be financially and technically prohibitive. Going forward, there is
still a place for 16S rRNA gene sequencing. Although sequencing
technologies will evolve to capture longer and more high quality data,
there will likely always be a need for characterizing the taxonomic
diversity of microbial communities. With this in mind, there will always
be a place for tools like mothur that can analyze amplicon sequence
data.

Of course this does not mean that such tools will remain static. We see
three key areas that we will continue to help the field to move forward.
First, just as we adapted through the transitions from Sanger to 454 to
MiSeq and PacBio sequencing platforms (60--62), we must learn whether
data from Oxford Nanopore can be an alternative sequencing approach that
generates sequence data that is the same quality as existing approaches;
thus far, the approach has significant shortcomings for sequencing 16S
rRNA gene sequences (74). As with the earlier platforms, we must better
understand its error profile so that sequencing errors can be corrected.
We have learned that moving forward requires that we maintain or improve
sequence quality. No doubt, datasets and read lengths will improve, but
these advances should not be made at the cost of data quality. Second,
with these improvements, we will need to continue to improve our
algorithms. We have already seen that attempts to use low quality MiSeq
and HiSeq data causes computational problems leading to the creation of
open and closed reference clustering methods (75, 76). Unfortunately,
comparative analyses showed that these methods fail relative to \emph{de
novo} clustering methods (57). More work is needed to improve
reference-based clustering methods so that larger datasets can be
analyzed without sacrificing quality. Finally, there are ongoing
controversies that need further exploration. These include the validity
and utility of amplicon sequence variants (77), the wisdom of removing
low frequency sequences (78), and methods of identifying and removing
contaminant 16S rRNA gene sequences (79, 80). With each of these areas
of development, the broader community can count on our same data-driven
approach to meeting evaluating these questions. It is common for
researchers to comment that they pick a specific method or deviate from
a suggestion because they ``like how the data look''. When pressed for
an objective definition of how they know the data look ``right'', they
go quiet. Through the use of mock communities and simulations where we
actually know what looks right and objective metrics of quality like the
MCC or sequencing error rates, we will continue to base recommendations
on data rather than a gut feeling.

\textbf{\emph{Conclusion.}} In the paper announcing mothur, we commented
that the relationship between 16S rRNA gene sequencing and analysis is
very much like the Red Queen in Lewis Carroll's book, \emph{Through the
Looking-Glass}. Although some disagreed with this analogy (81), I still
feel it is apt. The sequencing technology and rapacious appetite of
researchers continues to race on. At the same time, bioinformatics tools
must adapt to facilitate our research. I am confident that mothur will
be up to this exciting challenge. Beyond its utility for analyzing
amplicon sequence data, mothur's history provides lessons that are
helpful for other projects that hope to develop a long historical arc.
First, mothur is a product of its time. We have always sought to solve a
current need to the best of our ability with the tools we had at the
time. There are certainly caveats to any analysis of 16S rRNA gene
sequence data, but if we had waited until those caveats were resolved,
the field never would have progressed. Similarly, we made design choices
that we probably would not have made had we started the project today.
Second, as we have developed mothur, we have attempted to do so in a
data-driven approach where we compare multiple methods. It has not
merely been enough to propose a new method, we must show that it
meaningfully advances the field. Third, through our failures and
successes we have learned to focus on what mothur is good at and create
products separate from mothur when distinct needs arise. For example, we
have learned that mothur should not have a graphical interface or data
visualization tools. Instead, we will provide instructional materials to
teach users how to use the command line interface and other programs
like R for data visualization. Finally, mothur was born out of a need
for automating the analysis of large 16S rRNA gene sequence datasets. I
realized I had a set of skills to fill that need. It has been refreshing
to see the computational skills of the microbial ecology field grow over
the past two decades. Looking ahead, we must all take stock of the
challenges we face in microbial ecology and how our individual skills
and interests can address these challenges to turn them into
opportunities.

\hypertarget{acknowledgements}{%
\subsection{Acknowledgements}\label{acknowledgements}}

The development of mothur would not be possible without the
contributions of its many supporters, developers, and users. Although
not a complete list, I would be remiss if I did not express my gratitude
to Sarah Westcott and the other members of my research group, Jo
Handelsman, Mitch Sogin, Susan Huse, Vincent Young, Lita Proctor, Kendra
Mass, and Marcy Balunas for their unique contributions to the continued
development of mothur.

\newpage

\hypertarget{references}{%
\subsection{References}\label{references}}

\hypertarget{refs}{}
\leavevmode\hypertarget{ref-Lenski2017}{}%
1. \textbf{Lenski RE}. 2017. Experimental evolution and the dynamics of
adaptation and genome evolution in microbial populations. The ISME
Journal \textbf{11}:2181--2194.
doi:\href{https://doi.org/10.1038/ismej.2017.69}{10.1038/ismej.2017.69}.

\leavevmode\hypertarget{ref-Smith2018}{}%
2. \textbf{Smith DK}. 2018. From fundamental supramolecular chemistry to
self-assembled nanomaterials and medicines and back again how sam
inspired SAMul. Chemical Communications \textbf{54}:4743--4760.
doi:\href{https://doi.org/10.1039/c8cc01753k}{10.1039/c8cc01753k}.

\leavevmode\hypertarget{ref-Barbour2019}{}%
3. \textbf{Barbour AG}, \textbf{Benach JL}. 2019. Discovery of the lyme
disease agent. mBio \textbf{10}:e02166--19.
doi:\href{https://doi.org/10.1128/mbio.02166-19}{10.1128/mbio.02166-19}.

\leavevmode\hypertarget{ref-Colwell2014}{}%
4. \textbf{Colwell RK}, \textbf{Elsensohn JE}. 2014. EstimateS turns 20:
Statistical estimation of species richness and shared species from
samples, with non-parametric extrapolation. Ecography
\textbf{37}:609--613.
doi:\href{https://doi.org/10.1111/ecog.00814}{10.1111/ecog.00814}.

\leavevmode\hypertarget{ref-Glckner2017}{}%
5. \textbf{Glöckner FO}, \textbf{Yilmaz P}, \textbf{Quast C},
\textbf{Gerken J}, \textbf{Beccati A}, \textbf{Ciuprina A},
\textbf{Bruns G}, \textbf{Yarza P}, \textbf{Peplies J}, \textbf{Westram
R}, \textbf{Ludwig W}. 2017. 25 years of serving the community with
ribosomal RNA gene reference databases and tools. Journal of
Biotechnology \textbf{261}:169--176.
doi:\href{https://doi.org/10.1016/j.jbiotec.2017.06.1198}{10.1016/j.jbiotec.2017.06.1198}.

\leavevmode\hypertarget{ref-Casadevall2015}{}%
6. \textbf{Casadevall A}, \textbf{Fang FC}. 2015. (A)Historical science.
Infection and Immunity \textbf{83}:4460--4464.
doi:\href{https://doi.org/10.1128/iai.00921-15}{10.1128/iai.00921-15}.

\leavevmode\hypertarget{ref-Schloss2009a}{}%
7. \textbf{Schloss PD}, \textbf{Westcott SL}, \textbf{Ryabin T},
\textbf{Hall JR}, \textbf{Hartmann M}, \textbf{Hollister EB},
\textbf{Lesniewski RA}, \textbf{Oakley BB}, \textbf{Parks DH},
\textbf{Robinson CJ}, \textbf{Sahl JW}, \textbf{Stres B},
\textbf{Thallinger GG}, \textbf{Horn DJV}, \textbf{Weber CF}. 2009.
Introducing mothur: Open-source, platform-independent,
community-supported software for describing and comparing microbial
communities. Applied and Environmental Microbiology
\textbf{75}:7537--7541.
doi:\href{https://doi.org/10.1128/aem.01541-09}{10.1128/aem.01541-09}.

\leavevmode\hypertarget{ref-Hughes2001}{}%
8. \textbf{Hughes JB}, \textbf{Hellmann JJ}, \textbf{Ricketts TH},
\textbf{Bohannan BJM}. 2001. Counting the uncountable: Statistical
approaches to estimating microbial diversity. Applied and Environmental
Microbiology \textbf{67}:4399--4406.
doi:\href{https://doi.org/10.1128/aem.67.10.4399-4406.2001}{10.1128/aem.67.10.4399-4406.2001}.

\leavevmode\hypertarget{ref-Schloss2005}{}%
9. \textbf{Schloss PD}, \textbf{Handelsman J}. 2005. Introducing DOTUR,
a computer program for defining operational taxonomic units and
estimating species richness. Applied and Environmental Microbiology
\textbf{71}:1501--1506.
doi:\href{https://doi.org/10.1128/aem.71.3.1501-1506.2005}{10.1128/aem.71.3.1501-1506.2005}.

\leavevmode\hypertarget{ref-Schloss2004a}{}%
10. \textbf{Schloss PD}, \textbf{Larget BR}, \textbf{Handelsman J}.
2004. Integration of microbial ecology and statistics: A test to compare
gene libraries. Applied and Environmental Microbiology
\textbf{70}:5485--5492.
doi:\href{https://doi.org/10.1128/aem.70.9.5485-5492.2004}{10.1128/aem.70.9.5485-5492.2004}.

\leavevmode\hypertarget{ref-Schloss2006b}{}%
11. \textbf{Schloss PD}, \textbf{Handelsman J}. 2006. Introducing SONS,
a tool for operational taxonomic unit-based comparisons of microbial
community memberships and structures. Applied and Environmental
Microbiology \textbf{72}:6773--6779.
doi:\href{https://doi.org/10.1128/aem.00474-06}{10.1128/aem.00474-06}.

\leavevmode\hypertarget{ref-Schloss2006a}{}%
12. \textbf{Schloss PD}, \textbf{Handelsman J}. 2006. Introducing
TreeClimber, a test to compare microbial community structures. Applied
and Environmental Microbiology \textbf{72}:2379--2384.
doi:\href{https://doi.org/10.1128/aem.72.4.2379-2384.2006}{10.1128/aem.72.4.2379-2384.2006}.

\leavevmode\hypertarget{ref-Lozupone2005}{}%
13. \textbf{Lozupone C}, \textbf{Knight R}. 2005. UniFrac: A new
phylogenetic method for comparing microbial communities. Applied and
Environmental Microbiology \textbf{71}:8228--8235.
doi:\href{https://doi.org/10.1128/aem.71.12.8228-8235.2005}{10.1128/aem.71.12.8228-8235.2005}.

\leavevmode\hypertarget{ref-Lozupone2007}{}%
14. \textbf{Lozupone CA}, \textbf{Hamady M}, \textbf{Kelley ST},
\textbf{Knight R}. 2007. Quantitative and qualitative ~ diversity
measures lead to different insights into factors that structure
microbial communities. Applied and Environmental Microbiology
\textbf{73}:1576--1585.
doi:\href{https://doi.org/10.1128/aem.01996-06}{10.1128/aem.01996-06}.

\leavevmode\hypertarget{ref-Schloss2006c}{}%
15. \textbf{Schloss PD}, \textbf{Handelsman J}. 2006. Toward a census of
bacteria in soil. PLoS Computational Biology \textbf{2}:e92.
doi:\href{https://doi.org/10.1371/journal.pcbi.0020092}{10.1371/journal.pcbi.0020092}.

\leavevmode\hypertarget{ref-Schloss2004b}{}%
16. \textbf{Schloss PD}, \textbf{Handelsman J}. 2004. Status of the
microbial census. Microbiology and Molecular Biology Reviews
\textbf{68}:686--691.
doi:\href{https://doi.org/10.1128/mmbr.68.4.686-691.2004}{10.1128/mmbr.68.4.686-691.2004}.

\leavevmode\hypertarget{ref-Schloss2007}{}%
17. \textbf{Schloss PD}, \textbf{Handelsman J}. 2007. The last word:
Books as a statistical metaphor for microbial communities. Annual Review
of Microbiology \textbf{61}:23--34.

\leavevmode\hypertarget{ref-Zhou1996}{}%
18. \textbf{Zhou J}, \textbf{Bruns MA}, \textbf{Tiedje JM}. 1996. DNA
recovery from soils of diverse composition. Applied and Environmental
Microbiology \textbf{62}:316--322.

\leavevmode\hypertarget{ref-Suzuki1996}{}%
19. \textbf{Suzuki MT}, \textbf{Giovannoni SJ}. 1996. Bias caused by
template annealing in the amplification of mixtures of 16S rRNA genes by
pcr. Applied and Environmental Microbiology \textbf{62}:625--630.

\leavevmode\hypertarget{ref-Chandler1997}{}%
20. \textbf{Chandler DP}, \textbf{Fredrickson JK}, \textbf{Brockman FJ}.
1997. Effect of pcr template concentration on the composition and
distribution of total community 16S rDNA clone libraries. Molecular
Ecology \textbf{6}:475--482.

\leavevmode\hypertarget{ref-Polz1998}{}%
21. \textbf{Polz MF}, \textbf{Cavanaugh CM}. 1998. Bias in
template-to-product ratios in multitemplate pcr. Applied and
Environmental Microbiology \textbf{64}:3724--3730.

\leavevmode\hypertarget{ref-Wagner1994}{}%
22. \textbf{Wagner A}, \textbf{Blackstone N}, \textbf{Cartwright P},
\textbf{Dick M}, \textbf{Misof B}, \textbf{Snow P}, \textbf{Wagner GP},
\textbf{Bartels J}, \textbf{Murtha M}, \textbf{Pendleton J}. 1994.
Surveys of gene families using polymerase chain reaction: PCR selection
and pcr drift. Systematic Biology \textbf{43}:250--261.

\leavevmode\hypertarget{ref-Hansen1998}{}%
23. \textbf{Hansen MC}, \textbf{Tolker-Nielsen T}, \textbf{Givskov M},
\textbf{Molin S}. 1998. Biased 16S rDNA pcr amplification caused by
interference from dna flanking the template region. FEMS Microbiology
Ecology \textbf{26}:141--149.

\leavevmode\hypertarget{ref-Qiu2001}{}%
24. \textbf{Qiu X}, \textbf{Wu L}, \textbf{Huang H}, \textbf{McDonel
PE}, \textbf{Palumbo AV}, \textbf{Tiedje JM}, \textbf{Zhou J}. 2001.
Evaluation of pcr-generated chimeras, mutations, and heteroduplexes with
16S rRNA gene-based cloning. Applied and Environmental Microbiology
\textbf{67}:880--887.

\leavevmode\hypertarget{ref-Komatsoulis1997}{}%
25. \textbf{Komatsoulis GA}, \textbf{Waterman MS}. 1997. A new
computational method for detection of chimeric 16S rRNA artifacts
generated by pcr amplification from mixed bacterial populations. Applied
and Environmental Microbiology \textbf{63}:2338--2346.

\leavevmode\hypertarget{ref-Wang1997}{}%
26. \textbf{Wang G}, \textbf{Wang Y}. 1997. Frequency of formation of
chimeric molecules as a consequence of pcr coamplification of 16S rRNA
genes from mixed bacterial genomes. Applied and Environmental
Microbiology \textbf{63}:4645--4650.

\leavevmode\hypertarget{ref-Hugenholtz2003}{}%
27. \textbf{Hugenholtz P}, \textbf{Huber T}. 2003. Chimeric 16S rDNA
sequences of diverse origin are accumulating in the public databases.
International Journal of Systematic and Evolutionary Microbiology
\textbf{53}:289--293.

\leavevmode\hypertarget{ref-Bonfield1995}{}%
28. \textbf{Bonfield JK}, \textbf{Smith KF}, \textbf{Staden R}. 1995. A
new dna sequence assembly program. Nucleic Acids Research
\textbf{23}:4992--4999.

\leavevmode\hypertarget{ref-Thompson1994}{}%
29. \textbf{Thompson JD}, \textbf{Higgins DG}, \textbf{Gibson TJ}. 1994.
CLUSTAL w: Improving the sensitivity of progressive multiple sequence
alignment through sequence weighting, position-specific gap penalties
and weight matrix choice. Nucleic Acids Research \textbf{22}:4673--4680.
doi:\href{https://doi.org/10.1093/nar/22.22.4673}{10.1093/nar/22.22.4673}.

\leavevmode\hypertarget{ref-Ludwig2004}{}%
30. \textbf{Ludwig W}. 2004. ARB: A software environment for sequence
data. Nucleic Acids Research \textbf{32}:1363--1371.
doi:\href{https://doi.org/10.1093/nar/gkh293}{10.1093/nar/gkh293}.

\leavevmode\hypertarget{ref-Felsenstein1989}{}%
31. \textbf{Felsenstein J}. 1989. PHYLIP - phylogeny inference package.
Cladistics \textbf{5}:164--166.

\leavevmode\hypertarget{ref-Stackebrandt1994}{}%
32. \textbf{Stackebrandt E}, \textbf{Goebel BM}. 1994. Taxonomic note: A
place for DNA-DNA reassociation and 16S rRNA sequence analysis in the
present species definition in bacteriology. International Journal of
Systematic and Evolutionary Microbiology \textbf{44}:846--849.
doi:\href{https://doi.org/10.1099/00207713-44-4-846}{10.1099/00207713-44-4-846}.

\leavevmode\hypertarget{ref-Seguritan2001}{}%
33. \textbf{Seguritan V}, \textbf{Rohwer F}. 2001. FastGroup: A program
to dereplicate libraries of 16S rDNA sequences. BMC Bioinformatics
\textbf{2}:9.
doi:\href{https://doi.org/10.1186/1471-2105-2-9}{10.1186/1471-2105-2-9}.

\leavevmode\hypertarget{ref-McCaig1999}{}%
34. \textbf{McCaig AE}, \textbf{Glover LA}, \textbf{Prosser JI}. 1999.
Molecular analysis of bacterial community structure and diversity in
unimproved and improved upland grass pastures. Applied and Environmental
Microbiology \textbf{65}:1721--1730.

\leavevmode\hypertarget{ref-Pollock2018}{}%
35. \textbf{Pollock J}, \textbf{Glendinning L}, \textbf{Wisedchanwet T},
\textbf{Watson M}. 2018. The madness of microbiome: Attempting to find
consensus ``Best practice'' for 16S microbiome studies. Applied and
Environmental Microbiology \textbf{84}:e02627--17.
doi:\href{https://doi.org/10.1128/aem.02627-17}{10.1128/aem.02627-17}.

\leavevmode\hypertarget{ref-Singleton2001}{}%
36. \textbf{Singleton DR}, \textbf{Furlong MA}, \textbf{Rathbun SL},
\textbf{Whitman WB}. 2001. Quantitative comparisons of 16S rRNA gene
sequence libraries from environmental samples. Applied and Environmental
Microbiology \textbf{67}:4374--4376.
doi:\href{https://doi.org/10.1128/aem.67.9.4374-4376.2001}{10.1128/aem.67.9.4374-4376.2001}.

\leavevmode\hypertarget{ref-Carey2018}{}%
37. \textbf{Carey MA}, \textbf{Papin JA}. 2018. Ten simple rules for
biologists learning to program. PLOS Computational Biology
\textbf{14}:e1005871.
doi:\href{https://doi.org/10.1371/journal.pcbi.1005871}{10.1371/journal.pcbi.1005871}.

\leavevmode\hypertarget{ref-Sogin2006}{}%
38. \textbf{Sogin ML}, \textbf{Morrison HG}, \textbf{Huber JA},
\textbf{Welch DM}, \textbf{Huse SM}, \textbf{Neal PR}, \textbf{Arrieta
JM}, \textbf{Herndl GJ}. 2006. Microbial diversity in the deep sea and
the underexplored "rare biosphere". Proceedings of the National Academy
of Sciences \textbf{103}:12115--12120.
doi:\href{https://doi.org/10.1073/pnas.0605127103}{10.1073/pnas.0605127103}.

\leavevmode\hypertarget{ref-Cole2013}{}%
39. \textbf{Cole JR}, \textbf{Wang Q}, \textbf{Fish JA}, \textbf{Chai
B}, \textbf{McGarrell DM}, \textbf{Sun Y}, \textbf{Brown CT},
\textbf{Porras-Alfaro A}, \textbf{Kuske CR}, \textbf{Tiedje JM}. 2013.
Ribosomal database project: Data and tools for high throughput rRNA
analysis. Nucleic Acids Research \textbf{42}:D633--D642.
doi:\href{https://doi.org/10.1093/nar/gkt1244}{10.1093/nar/gkt1244}.

\leavevmode\hypertarget{ref-DeSantis2006a}{}%
40. \textbf{DeSantis TZ}, \textbf{Hugenholtz P}, \textbf{Larsen N},
\textbf{Rojas M}, \textbf{Brodie EL}, \textbf{Keller K}, \textbf{Huber
T}, \textbf{Dalevi D}, \textbf{Hu P}, \textbf{Andersen GL}. 2006.
Greengenes, a chimera-checked 16S rRNA gene database and workbench
compatible with ARB. Applied and Environmental Microbiology
\textbf{72}:5069--5072.
doi:\href{https://doi.org/10.1128/aem.03006-05}{10.1128/aem.03006-05}.

\leavevmode\hypertarget{ref-Yilmaz2013}{}%
41. \textbf{Yilmaz P}, \textbf{Parfrey LW}, \textbf{Yarza P},
\textbf{Gerken J}, \textbf{Pruesse E}, \textbf{Quast C}, \textbf{Schweer
T}, \textbf{Peplies J}, \textbf{Ludwig W}, \textbf{Glöckner FO}. 2013.
The SILVA and ``All-species living tree project (LTP)'' taxonomic
frameworks. Nucleic Acids Research \textbf{42}:D643--D648.
doi:\href{https://doi.org/10.1093/nar/gkt1209}{10.1093/nar/gkt1209}.

\leavevmode\hypertarget{ref-DeSantis2006b}{}%
42. \textbf{DeSantis TZ}, \textbf{Hugenholtz P}, \textbf{Keller K},
\textbf{Brodie EL}, \textbf{Larsen N}, \textbf{Piceno YM}, \textbf{Phan
R}, \textbf{Andersen GL}. 2006. NAST: A multiple sequence alignment
server for comparative analysis of 16S rRNA genes. Nucleic Acids
Research \textbf{34}:W394--W399.
doi:\href{https://doi.org/10.1093/nar/gkl244}{10.1093/nar/gkl244}.

\leavevmode\hypertarget{ref-Schloss2009b}{}%
43. \textbf{Schloss PD}. 2009. A high-throughput DNA sequence aligner
for microbial ecology studies. PLoS ONE \textbf{4}:e8230.
doi:\href{https://doi.org/10.1371/journal.pone.0008230}{10.1371/journal.pone.0008230}.

\leavevmode\hypertarget{ref-Schloss2010a}{}%
44. \textbf{Schloss PD}. 2010. The effects of alignment quality,
distance calculation method, sequence filtering, and region on the
analysis of 16S rRNA gene-based studies. PLoS Computational Biology
\textbf{6}:e1000844.
doi:\href{https://doi.org/10.1371/journal.pcbi.1000844}{10.1371/journal.pcbi.1000844}.

\leavevmode\hypertarget{ref-Wang2007}{}%
45. \textbf{Wang Q}, \textbf{Garrity GM}, \textbf{Tiedje JM},
\textbf{Cole JR}. 2007. Naive bayesian classifier for rapid assignment
of rRNA sequences into the new bacterial taxonomy. Applied and
Environmental Microbiology \textbf{73}:5261--5267.
doi:\href{https://doi.org/10.1128/aem.00062-07}{10.1128/aem.00062-07}.

\leavevmode\hypertarget{ref-Caporaso2010a}{}%
46. \textbf{Caporaso JG}, \textbf{Kuczynski J}, \textbf{Stombaugh J},
\textbf{Bittinger K}, \textbf{Bushman FD}, \textbf{Costello EK},
\textbf{Fierer N}, \textbf{Peña AG}, \textbf{Goodrich JK},
\textbf{Gordon JI}, \textbf{Huttley GA}, \textbf{Kelley ST},
\textbf{Knights D}, \textbf{Koenig JE}, \textbf{Ley RE},
\textbf{Lozupone CA}, \textbf{McDonald D}, \textbf{Muegge BD},
\textbf{Pirrung M}, \textbf{Reeder J}, \textbf{Sevinsky JR},
\textbf{Turnbaugh PJ}, \textbf{Walters WA}, \textbf{Widmann J},
\textbf{Yatsunenko T}, \textbf{Zaneveld J}, \textbf{Knight R}. 2010.
QIIME allows analysis of high-throughput community sequencing data.
Nature Methods \textbf{7}:335--336.
doi:\href{https://doi.org/10.1038/nmeth.f.303}{10.1038/nmeth.f.303}.

\leavevmode\hypertarget{ref-Quince2009}{}%
47. \textbf{Quince C}, \textbf{Lanzén A}, \textbf{Curtis TP},
\textbf{Davenport RJ}, \textbf{Hall N}, \textbf{Head IM}, \textbf{Read
LF}, \textbf{Sloan WT}. 2009. Accurate determination of microbial
diversity from 454 pyrosequencing data. Nature Methods
\textbf{6}:639--641.
doi:\href{https://doi.org/10.1038/nmeth.1361}{10.1038/nmeth.1361}.

\leavevmode\hypertarget{ref-Masella2012}{}%
48. \textbf{Masella AP}, \textbf{Bartram AK}, \textbf{Truszkowski JM},
\textbf{Brown DG}, \textbf{Neufeld JD}. 2012. PANDAseq: Paired-end
assembler for illumina sequences. BMC Bioinformatics \textbf{13}:31.
doi:\href{https://doi.org/10.1186/1471-2105-13-31}{10.1186/1471-2105-13-31}.

\leavevmode\hypertarget{ref-Callahan2016}{}%
49. \textbf{Callahan BJ}, \textbf{McMurdie PJ}, \textbf{Rosen MJ},
\textbf{Han AW}, \textbf{Johnson AJA}, \textbf{Holmes SP}. 2016. DADA2:
High-resolution sample inference from illumina amplicon data. Nature
Methods \textbf{13}:581--583.
doi:\href{https://doi.org/10.1038/nmeth.3869}{10.1038/nmeth.3869}.

\leavevmode\hypertarget{ref-Edgar2011}{}%
50. \textbf{Edgar RC}, \textbf{Haas BJ}, \textbf{Clemente JC},
\textbf{Quince C}, \textbf{Knight R}. 2011. UCHIME improves sensitivity
and speed of chimera detection. Bioinformatics \textbf{27}:2194--2200.
doi:\href{https://doi.org/10.1093/bioinformatics/btr381}{10.1093/bioinformatics/btr381}.

\leavevmode\hypertarget{ref-Haas2011}{}%
51. \textbf{Haas BJ}, \textbf{Gevers D}, \textbf{Earl AM},
\textbf{Feldgarden M}, \textbf{Ward DV}, \textbf{Giannoukos G},
\textbf{Ciulla D}, \textbf{Tabbaa D}, \textbf{Highlander SK},
\textbf{Sodergren E}, \textbf{Methe B}, \textbf{DeSantis TZ},
\textbf{Petrosino JF}, \textbf{Knight R}, \textbf{and BWB}. 2011.
Chimeric 16S rRNA sequence formation and detection in Sanger and
454-pyrosequenced PCR amplicons. Genome Research \textbf{21}:494--504.
doi:\href{https://doi.org/10.1101/gr.112730.110}{10.1101/gr.112730.110}.

\leavevmode\hypertarget{ref-Quince2011}{}%
52. \textbf{Quince C}, \textbf{Lanzen A}, \textbf{Davenport RJ},
\textbf{Turnbaugh PJ}. 2011. Removing noise from pyrosequenced
amplicons. BMC Bioinformatics \textbf{12}:38.
doi:\href{https://doi.org/10.1186/1471-2105-12-38}{10.1186/1471-2105-12-38}.

\leavevmode\hypertarget{ref-Edgar2010}{}%
53. \textbf{Edgar RC}. 2010. Search and clustering orders of magnitude
faster than BLAST. Bioinformatics \textbf{26}:2460--2461.
doi:\href{https://doi.org/10.1093/bioinformatics/btq461}{10.1093/bioinformatics/btq461}.

\leavevmode\hypertarget{ref-Rognes2016}{}%
54. \textbf{Rognes T}, \textbf{Flouri T}, \textbf{Nichols B},
\textbf{Quince C}, \textbf{Mahé F}. 2016. VSEARCH: A versatile open
source tool for metagenomics. PeerJ \textbf{4}:e2584.
doi:\href{https://doi.org/10.7717/peerj.2584}{10.7717/peerj.2584}.

\leavevmode\hypertarget{ref-Mah2015}{}%
55. \textbf{Mahé F}, \textbf{Rognes T}, \textbf{Quince C},
\textbf{Vargas C de}, \textbf{Dunthorn M}. 2015. Swarm v2:
Highly-scalable and high-resolution amplicon clustering. PeerJ
\textbf{3}:e1420.
doi:\href{https://doi.org/10.7717/peerj.1420}{10.7717/peerj.1420}.

\leavevmode\hypertarget{ref-Schloss2011a}{}%
56. \textbf{Schloss PD}, \textbf{Westcott SL}. 2011. Assessing and
improving methods used in operational taxonomic unit-based approaches
for 16S rRNA gene sequence analysis. Applied and Environmental
Microbiology \textbf{77}:3219--3226.
doi:\href{https://doi.org/10.1128/aem.02810-10}{10.1128/aem.02810-10}.

\leavevmode\hypertarget{ref-Westcott2015}{}%
57. \textbf{Westcott SL}, \textbf{Schloss PD}. 2015. De novo clustering
methods outperform reference-based methods for assigning 16S rRNA gene
sequences to operational taxonomic units. PeerJ \textbf{3}:e1487.
doi:\href{https://doi.org/10.7717/peerj.1487}{10.7717/peerj.1487}.

\leavevmode\hypertarget{ref-Schloss2016a}{}%
58. \textbf{Schloss PD}. 2016. Application of a database-independent
approach to assess the quality of operational taxonomic unit picking
methods. mSystems \textbf{1}:e00027--16.
doi:\href{https://doi.org/10.1128/msystems.00027-16}{10.1128/msystems.00027-16}.

\leavevmode\hypertarget{ref-Westcott2017}{}%
59. \textbf{Westcott SL}, \textbf{Schloss PD}. 2017. OptiClust, an
improved method for assigning amplicon-based sequence data to
operational taxonomic units. mSphere \textbf{2}:e00073--17.
doi:\href{https://doi.org/10.1128/mspheredirect.00073-17}{10.1128/mspheredirect.00073-17}.

\leavevmode\hypertarget{ref-Kozich2013}{}%
60. \textbf{Kozich JJ}, \textbf{Westcott SL}, \textbf{Baxter NT},
\textbf{Highlander SK}, \textbf{Schloss PD}. 2013. Development of a
dual-index sequencing strategy and curation pipeline for analyzing
amplicon sequence data on the MiSeq illumina sequencing platform.
Applied and Environmental Microbiology \textbf{79}:5112--5120.
doi:\href{https://doi.org/10.1128/aem.01043-13}{10.1128/aem.01043-13}.

\leavevmode\hypertarget{ref-Schloss2011b}{}%
61. \textbf{Schloss PD}, \textbf{Gevers D}, \textbf{Westcott SL}. 2011.
Reducing the effects of PCR amplification and sequencing artifacts on
16S rRNA-based studies. PLoS ONE \textbf{6}:e27310.
doi:\href{https://doi.org/10.1371/journal.pone.0027310}{10.1371/journal.pone.0027310}.

\leavevmode\hypertarget{ref-Schloss2016b}{}%
62. \textbf{Schloss PD}, \textbf{Jenior ML}, \textbf{Koumpouras CC},
\textbf{Westcott SL}, \textbf{Highlander SK}. 2016. Sequencing 16S rRNA
gene fragments using the PacBio SMRT DNA sequencing system. PeerJ
\textbf{4}:e1869.
doi:\href{https://doi.org/10.7717/peerj.1869}{10.7717/peerj.1869}.

\leavevmode\hypertarget{ref-Schloss2012a}{}%
63. \textbf{Schloss PD}. 2012. Secondary structure improves OTU
assignments of 16S rRNA gene sequences. The ISME Journal
\textbf{7}:457--460.
doi:\href{https://doi.org/10.1038/ismej.2012.102}{10.1038/ismej.2012.102}.

\leavevmode\hypertarget{ref-Schloss2008}{}%
64. \textbf{Schloss PD}. 2008. Evaluating different approaches that test
whether microbial communities have the same structure. The ISME Journal
\textbf{2}:265--275.
doi:\href{https://doi.org/10.1038/ismej.2008.5}{10.1038/ismej.2008.5}.

\leavevmode\hypertarget{ref-Schloss2018a}{}%
65. \textbf{Schloss PD}. 2018. The riffomonas reproducible research
tutorial series. Journal of Open Source Education \textbf{1}:13.
doi:\href{https://doi.org/10.21105/jose.00013}{10.21105/jose.00013}.

\leavevmode\hypertarget{ref-Eddelbuettel2011}{}%
66. \textbf{Eddelbuettel D}, \textbf{François R}. 2011. Rcpp: Seamless R
and C++ integration. Journal of Statistical Software \textbf{40}:1--18.
doi:\href{https://doi.org/10.18637/jss.v040.i08}{10.18637/jss.v040.i08}.

\leavevmode\hypertarget{ref-VzquezBaeza2013}{}%
67. \textbf{Vázquez-Baeza Y}, \textbf{Pirrung M}, \textbf{Gonzalez A},
\textbf{Knight R}. 2013. EMPeror: A tool for visualizing high-throughput
microbial community data. GigaScience \textbf{2}:16.
doi:\href{https://doi.org/10.1186/2047-217x-2-16}{10.1186/2047-217x-2-16}.

\leavevmode\hypertarget{ref-Bolyen2019}{}%
68. \textbf{Bolyen E}, \textbf{Rideout JR}, \textbf{Dillon MR},
\textbf{Bokulich NA}, \textbf{Abnet CC}, \textbf{Al-Ghalith GA},
\textbf{Alexander H}, \textbf{Alm EJ}, \textbf{Arumugam M},
\textbf{Asnicar F}, \textbf{Bai Y}, \textbf{Bisanz JE},
\textbf{Bittinger K}, \textbf{Brejnrod A}, \textbf{Brislawn CJ},
\textbf{Brown CT}, \textbf{Callahan BJ}, \textbf{Caraballo-Rodrı́guez
AM}, \textbf{Chase J}, \textbf{Cope EK}, \textbf{Silva RD},
\textbf{Diener C}, \textbf{Dorrestein PC}, \textbf{Douglas GM},
\textbf{Durall DM}, \textbf{Duvallet C}, \textbf{Edwardson CF},
\textbf{Ernst M}, \textbf{Estaki M}, \textbf{Fouquier J},
\textbf{Gauglitz JM}, \textbf{Gibbons SM}, \textbf{Gibson DL},
\textbf{Gonzalez A}, \textbf{Gorlick K}, \textbf{Guo J},
\textbf{Hillmann B}, \textbf{Holmes S}, \textbf{Holste H},
\textbf{Huttenhower C}, \textbf{Huttley GA}, \textbf{Janssen S},
\textbf{Jarmusch AK}, \textbf{Jiang L}, \textbf{Kaehler BD},
\textbf{Kang KB}, \textbf{Keefe CR}, \textbf{Keim P}, \textbf{Kelley
ST}, \textbf{Knights D}, \textbf{Koester I}, \textbf{Kosciolek T},
\textbf{Kreps J}, \textbf{Langille MGI}, \textbf{Lee J}, \textbf{Ley R},
\textbf{Liu Y-X}, \textbf{Loftfield E}, \textbf{Lozupone C},
\textbf{Maher M}, \textbf{Marotz C}, \textbf{Martin BD},
\textbf{McDonald D}, \textbf{McIver LJ}, \textbf{Melnik AV},
\textbf{Metcalf JL}, \textbf{Morgan SC}, \textbf{Morton JT},
\textbf{Naimey AT}, \textbf{Navas-Molina JA}, \textbf{Nothias LF},
\textbf{Orchanian SB}, \textbf{Pearson T}, \textbf{Peoples SL},
\textbf{Petras D}, \textbf{Preuss ML}, \textbf{Pruesse E},
\textbf{Rasmussen LB}, \textbf{Rivers A}, \textbf{Robeson MS},
\textbf{Rosenthal P}, \textbf{Segata N}, \textbf{Shaffer M},
\textbf{Shiffer A}, \textbf{Sinha R}, \textbf{Song SJ}, \textbf{Spear
JR}, \textbf{Swafford AD}, \textbf{Thompson LR}, \textbf{Torres PJ},
\textbf{Trinh P}, \textbf{Tripathi A}, \textbf{Turnbaugh PJ},
\textbf{Ul-Hasan S}, \textbf{Hooft JJJ van der}, \textbf{Vargas F},
\textbf{Vázquez-Baeza Y}, \textbf{Vogtmann E}, \textbf{Hippel M von},
\textbf{Walters W}, \textbf{Wan Y}, \textbf{Wang M}, \textbf{Warren J},
\textbf{Weber KC}, \textbf{Williamson CHD}, \textbf{Willis AD},
\textbf{Xu ZZ}, \textbf{Zaneveld JR}, \textbf{Zhang Y}, \textbf{Zhu Q},
\textbf{Knight R}, \textbf{Caporaso JG}. 2019. Reproducible,
interactive, scalable and extensible microbiome data science using QIIME
2. Nature Biotechnology \textbf{37}:852--857.
doi:\href{https://doi.org/10.1038/s41587-019-0209-9}{10.1038/s41587-019-0209-9}.

\leavevmode\hypertarget{ref-Gonzalez2018}{}%
69. \textbf{Gonzalez A}, \textbf{Navas-Molina JA}, \textbf{Kosciolek T},
\textbf{McDonald D}, \textbf{Vázquez-Baeza Y}, \textbf{Ackermann G},
\textbf{DeReus J}, \textbf{Janssen S}, \textbf{Swafford AD},
\textbf{Orchanian SB}, \textbf{Sanders JG}, \textbf{Shorenstein J},
\textbf{Holste H}, \textbf{Petrus S}, \textbf{Robbins-Pianka A},
\textbf{Brislawn CJ}, \textbf{Wang M}, \textbf{Rideout JR},
\textbf{Bolyen E}, \textbf{Dillon M}, \textbf{Caporaso JG},
\textbf{Dorrestein PC}, \textbf{Knight R}. 2018. Qiita: Rapid,
web-enabled microbiome meta-analysis. Nature Methods
\textbf{15}:796--798.
doi:\href{https://doi.org/10.1038/s41592-018-0141-9}{10.1038/s41592-018-0141-9}.

\leavevmode\hypertarget{ref-Schloss2018b}{}%
70. \textbf{Schloss PD}. 2018. Identifying and overcoming threats to
reproducibility, replicability, robustness, and generalizability in
microbiome research. mBio \textbf{9}:e00525--18.
doi:\href{https://doi.org/10.1128/mbio.00525-18}{10.1128/mbio.00525-18}.

\leavevmode\hypertarget{ref-Paszke2017}{}%
71. \textbf{Paszke A}, \textbf{Gross S}, \textbf{Chintala S},
\textbf{Chanan G}, \textbf{Yang E}, \textbf{DeVito Z}, \textbf{Lin Z},
\textbf{Desmaison A}, \textbf{Antiga L}, \textbf{Lerer A}. 2017.
Automatic differentiation in PyTorch. \emph{In} NIPS autodiff workshop.

\leavevmode\hypertarget{ref-Kuhn2008}{}%
72. \textbf{Kuhn M}. 2008. Building predictive models in R using the
caret package. Journal of Statistical Software, Articles
\textbf{28}:1--26.
doi:\href{https://doi.org/10.18637/jss.v028.i05}{10.18637/jss.v028.i05}.

\leavevmode\hypertarget{ref-Baxter2016}{}%
73. \textbf{Baxter NT}, \textbf{Ruffin MT}, \textbf{Rogers MAM},
\textbf{Schloss PD}. 2016. Microbiota-based model improves the
sensitivity of fecal immunochemical test for detecting colonic lesions.
Genome Medicine \textbf{8}:37.
doi:\href{https://doi.org/10.1186/s13073-016-0290-3}{10.1186/s13073-016-0290-3}.

\leavevmode\hypertarget{ref-Calus2018}{}%
74. \textbf{Calus ST}, \textbf{Ijaz UZ}, \textbf{Pinto AJ}. 2018.
NanoAmpli-seq: A workflow for amplicon sequencing for mixed microbial
communities on the nanopore sequencing platform. GigaScience
\textbf{7}:12.
doi:\href{https://doi.org/10.1093/gigascience/giy140}{10.1093/gigascience/giy140}.

\leavevmode\hypertarget{ref-NavasMolina2013}{}%
75. \textbf{Navas-Molina JA}, \textbf{Peralta-Sánchez JM},
\textbf{González A}, \textbf{McMurdie PJ}, \textbf{Vázquez-Baeza Y},
\textbf{Xu Z}, \textbf{Ursell LK}, \textbf{Lauber C}, \textbf{Zhou H},
\textbf{Song SJ}, \textbf{Huntley J}, \textbf{Ackermann GL},
\textbf{Berg-Lyons D}, \textbf{Holmes S}, \textbf{Caporaso JG},
\textbf{Knight R}. 2013. Advancing our understanding of the human
microbiome using QIIME, pp. 371--444. \emph{In} Methods in Enzymology.
Elsevier.

\leavevmode\hypertarget{ref-Rideout2014}{}%
76. \textbf{Rideout JR}, \textbf{He Y}, \textbf{Navas-Molina JA},
\textbf{Walters WA}, \textbf{Ursell LK}, \textbf{Gibbons SM},
\textbf{Chase J}, \textbf{McDonald D}, \textbf{Gonzalez A},
\textbf{Robbins-Pianka A}, \textbf{Clemente JC}, \textbf{Gilbert JA},
\textbf{Huse SM}, \textbf{Zhou H-W}, \textbf{Knight R}, \textbf{Caporaso
JG}. 2014. Subsampled open-reference clustering creates consistent,
comprehensive OTU definitions and scales to billions of sequences. PeerJ
\textbf{2}:e545.
doi:\href{https://doi.org/10.7717/peerj.545}{10.7717/peerj.545}.

\leavevmode\hypertarget{ref-Callahan2017}{}%
77. \textbf{Callahan BJ}, \textbf{McMurdie PJ}, \textbf{Holmes SP}.
2017. Exact sequence variants should replace operational taxonomic units
in marker-gene data analysis. The ISME Journal \textbf{11}:2639--2643.
doi:\href{https://doi.org/10.1038/ismej.2017.119}{10.1038/ismej.2017.119}.

\leavevmode\hypertarget{ref-Bokulich2012}{}%
78. \textbf{Bokulich NA}, \textbf{Subramanian S}, \textbf{Faith JJ},
\textbf{Gevers D}, \textbf{Gordon JI}, \textbf{Knight R}, \textbf{Mills
DA}, \textbf{Caporaso JG}. 2012. Quality-filtering vastly improves
diversity estimates from illumina amplicon sequencing. Nature Methods
\textbf{10}:57--59.
doi:\href{https://doi.org/10.1038/nmeth.2276}{10.1038/nmeth.2276}.

\leavevmode\hypertarget{ref-Salter2014}{}%
79. \textbf{Salter SJ}, \textbf{Cox MJ}, \textbf{Turek EM},
\textbf{Calus ST}, \textbf{Cookson WO}, \textbf{Moffatt MF},
\textbf{Turner P}, \textbf{Parkhill J}, \textbf{Loman NJ},
\textbf{Walker AW}. 2014. Reagent and laboratory contamination can
critically impact sequence-based microbiome analyses. BMC Biology
\textbf{12}:87.
doi:\href{https://doi.org/10.1186/s12915-014-0087-z}{10.1186/s12915-014-0087-z}.

\leavevmode\hypertarget{ref-Davis2018}{}%
80. \textbf{Davis NM}, \textbf{Proctor DM}, \textbf{Holmes SP},
\textbf{Relman DA}, \textbf{Callahan BJ}. 2018. Simple statistical
identification and removal of contaminant sequences in marker-gene and
metagenomics data. Microbiome \textbf{6}:1.
doi:\href{https://doi.org/10.1186/s40168-018-0605-2}{10.1186/s40168-018-0605-2}.

\leavevmode\hypertarget{ref-Caporaso2010b}{}%
81. \textbf{Caporaso JG}, \textbf{Lauber CL}, \textbf{Walters WA},
\textbf{Berg-Lyons D}, \textbf{Lozupone CA}, \textbf{Turnbaugh PJ},
\textbf{Fierer N}, \textbf{Knight R}. 2010. Global patterns of 16S rRNA
diversity at a depth of millions of sequences per sample. Proceedings of
the National Academy of Sciences \textbf{108}:4516--4522.
doi:\href{https://doi.org/10.1073/pnas.1000080107}{10.1073/pnas.1000080107}.

\newpage

\textbf{Figure 1. Caption caption caption.} Footnotes footnotes
footnotes.

\begin{itemize}
\tightlist
\item
  Timeline of mothur/QIIME/etc citations
\item
  Timeline of SRA deposits / 16S/microbiome in PubMed
\end{itemize}

\textbf{Figure 2. Caption caption caption.} Footnotes footnotes
footnotes.

\begin{itemize}
\tightlist
\item
  Screenshot of mothur welcoming page
\end{itemize}

\textbf{Figure 3. Caption caption caption.} Footnotes footnotes
footnotes.

\begin{itemize}
\tightlist
\item
  Screenshot of mothur homepage
\end{itemize}


\end{document}
